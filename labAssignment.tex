\documentclass{article}

% Packages
\usepackage[utf8]{inputenc} % Input encoding
\usepackage[T1]{fontenc} % Font encoding
\usepackage{lipsum} % Dummy text package
\usepackage{graphicx}
\usepackage{listings} % For code listings
\usepackage{xcolor} % For custom colors
\usepackage [colorlinks=true, linkcolor=blue]{hyperref}
\usepackage[margin=1 cm]{geometry} % Adjust the margin values as needed
\usepackage{float}

% Define code listing style for assembly
\lstdefinestyle{asmStyle}{
  language=[x86masm]Assembler,
  basicstyle=\ttfamily,
  keywordstyle=\color{blue},
  commentstyle=\color{green!60!black},
  stringstyle=\color{red},
  showstringspaces=false,
  breaklines=true,
  frame=single,
  numbers=none,
  numberstyle=\tiny,
  numbersep=5pt,
  tabsize=2
}

% Title and author information



 

\title{
\begin{center}
\includegraphics[scale=0.1]{logo.png}
\end{center}

Department of Computer Science \& Engineering \\
\large{
University of Chittagong \\
course:Microprocessor \\
course code: CSE-416 \\
Final Lab Assignment
}
\author{Submitted to: \\
Md. Mahbubul Islam \\
Assistant Professor \\
Department of Computer Science and Engineering \\
University of Chittagong \\
\\
Submitted by: \\
Kazi omar sharif \\
20701015 \\
Department of Computer Science and Engineering \\
University of Chittagong} 

\date{\today} % Use \date{} for no date or specify a specific date

}

% Document content
\begin{document}

\maketitle

\newpage


\section{Write a program to display string “Computer Science and Engineering”} 

\begin{lstlisting}[style=asmStyle]
 .model small
.stack 100h
.data 
msg db 3
msg1 db ?  
hello DB "computer science and engineering$"
    

.code

main proc 
    mov ax,@data
    mov ds,ax 
      
      
    LEA dx , hello  
    mov ah,09h
    int 21h 
    
    
  
    .exit
    mov ax,4ch
    int 21h
    
    main endp

\end{lstlisting}


\section{Write a program to multiply 2 numbers (16-bit data) for 8086.} 

\begin{lstlisting}[style=asmStyle]
.model small
.stack 100h

.data
num1 dw ?
num2 dw ? 
result dw ? 
 count dw 0
p db 'result:$'   


.code  

  read_fun proc near  
    
   xor bx,bx
   xor cx,cx
   
   READ: 
   cmp cx,2
   JE END_READ  
   
   mov ah,01h
   int 21h
   JE READ
   
   
    cmp AL,10h 
    JE READ
    
    cmp AL,13h
    JE END_READ
    inc cx
    sub al,48 
    
    
    mov ah,00h
    push ax
    mov ax,bx
    mov bx,10
    mul bx
    mov bx,ax
    pop ax
    add bx,ax
    jmp READ
    
    END_READ:
   
    mov ah,2
    mov dl,32 
    int 21h
   
    
     ret
     
     read_fun endp 



write_fun proc near
     
     mov ax,bx
     mov dx,0 
       
     
     write:
    
     mov bx,10
     DIV bx 
     mov cx,ax
     
     add dl,30h 
      mov dh,0  
     push dx
    add count,1
     
     
     mov ax,cx
     mov dx,0
     
     test ax,ax
     jnz write  
          
      mov cx,count
      mov ah,2
      print:
      pop dx
      int 21h
      loop print
     
     ret
 write_fun ENDP 
     






main proc
   mov ax, @data
   mov ds, ax

   
   call read_fun  
   
   mov num1,bx
   xor bx,bx
   call read_fun 
   mov num2,bx
   
   xor bx,bx
   
   
   xor ax,ax
   mov ax,num1
   mul num2
   
   mov result ,ax
   mov bx,ax 
   
   mov ah,9
   LEA dx,p
   int 21h
   
   call write_fun 


   mov ah, 4Ch
   int 21h
main endp

end main


\end{lstlisting}

\section{Sum of series of 10 numbers and store result in memory location named Total.} 

\begin{lstlisting}[style=asmStyle]
.model small
.stack 100h

.data
Total dw ?          ; Memory location to store the result

.code
main proc
    mov ax, @data
    mov ds, ax      ; Initialize data segment

    mov cx, 10      ; Number of iterations
    mov ax, 0       ; Initialize the sum to 0

sumLoop:
    add ax, cx      ; Add the current value of CX to the sum
         
    loop sumLoop    ; Loop for all iterations

    mov [Total], ax ; Store the sum in the Total memory location

    
    mov ah, 02h     ; DOS function to display a character
    mov dl, 'T'     ; Display 'T'
    int 21h

    mov dl, 'o'     ; Display 'o'
    int 21h

    mov dl, 't'     ; Display 't'
    int 21h

    mov dl, 'a'     ; Display 'a'
    int 21h

    mov dl, 'l'     ; Display 'l'
    int 21h

    mov dl, ':'     ; Display ':'
    int 21h

    mov ax, [Total] ; Load the sum from the Total memory location
    call display16bitDecimal   ; Display the sum as a 16-bit decimal number

    mov ah, 4Ch     ; DOS function to exit the program
    int 21h         ; Call DOS

main endp

display16bitDecimal proc
    push bx
    push ax

    mov bx, 10      ; Base 10 (decimal)
    xor cx, cx      ; Initialize digit counter

digitLoop:
    xor dx, dx      ; Clear DX
    div bx          ; Divide AX by BX, quotient in AX, remainder in DX
    push dx         ; Push the digit to the stack
    inc cx          ; Increment digit counter
    test ax, ax     ; Check if quotient is zero
    jnz digitLoop   ; If not zero, continue loop

displayLoop:
    pop dx          ; Pop digit from the stack
    add dl, '0'     ; Convert remainder to ASCII character
    mov ah, 02h     ; DOS function to display a character
    int 21h         ; Call DOS

    loop displayLoop ; Loop until all digits are displayed

    pop ax
    pop bx
    ret

end main


\end{lstlisting}


\section{Write a program to find largest number in a block of data. The length of the block
is 0A. Store the maximum in location Result.} 

\begin{lstlisting}[style=asmStyle]
.model small
.stack 100h
  
  
  
.data   

max dw ?    
arr dw 10 dup(0) 
count dw 0 
track dw 0   
p db 'maximum:$' 

.code

main proc       
    
    mov ax, @data
    mov ds, ax  
    
      
    xor bx,bx
    mov cx,10  
    mov si, offset arr   
    
input: 
 
    mov track,cx
    call read_fun   
    mov [si],bx
    add si,2  
    mov cx,track 
    
loop input 
    
    
    
   
    mov ax, 0
    mov cx, 10 
    mov si,offset arr
    
loopArray: 

    mov bx, [si]    
    cmp bx,ax  ; cmp destination, source
    jle update_max
    mov ax, bx  
    
    
    update_max:
    add si, 2
loop loopArray    


   
   mov bx,0 
   mov bx, ax 
   mov max,bx  
  
      
      
   mov ah,2
   mov dl,10
   int 21h
   mov dl,13
   int 21h 
   
   
   mov ah,9
   LEA dx,p
   int 21h    
        
 call write_fun 
   
  
    
    
  .exit
  mov ah, 4ch
  int 21h
  
    
    
    

main endp

    
    
 read_fun proc near  
    
   xor bx,bx
   xor cx,cx
   
READ: 
     cmp cx,2
     JE END_READ  
   
     mov ah,01h
     int 21h
     JE READ
   
   
     cmp AL,10h 
     JE READ
    
     cmp AL,13h
     JE END_READ
     inc cx
     sub al,48 
    
    
    mov ah,00h
    push ax
    mov ax,bx
    mov bx,10
    mul bx
    mov bx,ax
    pop ax
    add bx,ax
    jmp READ
    
END_READ:
   
    mov ah,2
    mov dl,32 
    int 21h
   
    
     ret
read_fun endp 


write_fun proc near
     
     mov ax,bx
     mov dx,0 
       
     
write:
    
     mov bx,10
     DIV bx 
     mov cx,ax
     
     add dl,30h 
     mov dh,0  
     push dx
     add count,1
     
     
     mov ax,cx
     mov dx,0
     
     test ax,ax
     jnz write  
          
      mov cx,count
      mov ah,2  
      
print:
      pop dx
      int 21h 
      loop print
     
     ret
 write_fun ENDP 


\end{lstlisting}


\section{Find number of times letter ‘e’ exist in the string ‘exercise’, Store the count at
memory Answer.} 

\begin{lstlisting}[style=asmStyle]
.model small
.stack 100h

.data
answer dw 0
string db "exercise$"

.code
main proc
    mov ax, @data
    mov ds, ax
    
    mov cx, 0
    mov SI, offset string

    check: 
    mov al, [SI]
    cmp al, '$'
    je finish

    cmp al, 'e'
    jne not_e
    inc answer

    not_e:
    inc SI
    jmp check

    finish: 
    mov Ax, answer  
    
    call printNumber

    mov ah, 4Ch
    int 21h

printNumber proc
    push ax
    push bx
    push cx
    push dx

    mov cx, 0
    mov bx, 10

    nextDigit:
    xor dx, dx
    div bx
    push dx
    inc cx
    test ax, ax
    jnz nextDigit

    printLoop:
    pop dx
    add dl, 30h
    mov ah, 02h
    int 21h
    loop printLoop

    pop dx
    pop cx
    pop bx
    pop ax
    ret

end main

\end{lstlisting}


\section{Write an assembly language program to find factorial of number.} 

\begin{lstlisting}[style=asmStyle]

.MODEL SMALL
.STACK 100H

.DATA
    num1 DB ?
    num2 DB ?
    N DW ?
    Ans DW 1 

.CODE
    MAIN PROC  
        
        mov Ax,@Data
        mov DS,AX
        
        XOR AX,AX
        MOV AX,0000H
        MOV AH,1
        INT 21H
        
        MOV AH,00H
        MOV N,AX
        SUB N,48
        
        
        MOV CX,N
        MOV AX,1
        
        FACT:
        MUL CX
        LOOP FACT
        
        MOV ANS,AX
        MOV BX,ANS
       
       ;ADD BX,48
       
       .EXIT
       MOV AH,4CH
       INT 21H
       MAIN ENDP
    
    END MAIN
        
 


\end{lstlisting}



\section{Write an assembly language program to count number of vowels in a given
string.} 

\begin{lstlisting}[style=asmStyle]
.MODEL SMALL
.STACK 100H
.DATA
A DB 'STRING: $'
C DB 10,13,'TOTAL VOWAL:$'
.CODE
MAIN PROC
    MOV AX,@DATA
    MOV DS,AX
    
    MOV AH,9
    LEA DX,A
    INT 21H
    
    MOV BL,0
    MOV CL,0
    
    MOV AH,1
    INT 21H

WHILE:
    CMP AL,0DH ; 
    JE ENDL
    
    CMP AL,'A'
    JE OK
    CMP AL,'a'
    JE OK
    CMP AL,'E'
    JE OK
    CMP AL,'e'
    JE OK
    CMP AL,'I'
    JE OK
    CMP AL,'i'
    JE OK
    CMP AL,'O'
    JE OK
    CMP AL,'o'
    JE OK
    CMP AL,'U'
    JE OK
    CMP AL,'u'
    JE OK
    
    UP:
    INC BL
    INT 21H
    JMP WHILE      
    
    
OK:
    INC CL
    JMP UP
    
ENDL:    
    MOV AH,9
    LEA DX,C
    INT 21H
    
    MOV AH,2
    MOV DL,CL
    ADD DL,30H
    INT 21H
    
    ; EXIT PROGRAM
    MOV AH, 0
INT 20H
    
    MAIN ENDP
END MAIN

\end{lstlisting}



\section{Write an 8086 ALP which will input the user name from the keyboard. If the user
is ‘student’ it will output “The username is validi” else it will output “Invalid user
name”.} 

\begin{lstlisting}[style=asmStyle]
.MODEL SMALL
.DATA
      VAR1 DB 100 DUP('$')
      NUM1 DW 7
      MSG1 DB "student$"
      MSG2 DB 10, 13, "THE USERNAME IS VALID$"
      MSG3 DB 10, 13, "INVALID USER NAME$" 
.CODE
MAIN PROC   
      MOV AX,  @DATA
      MOV DS, AX
      
      MOV SI,OFFSET VAR1
      MOV CX, NUM1 
      
      LOOP1:
      MOV AH,1
      INT 21H
      CMP AL,13
      JNE INCREMENT
      
      PROGRAM:
      INC SI
      DEC CX
      JNZ LOOP1
      
      MOV CX,NUM1
      MOV SI,OFFSET VAR1  
      MOV DI,OFFSET MSG1
      
      
      IS_EQUAL:
      MOV BX,[SI]
      CMP BX,[DI]
      JNE INVALID
      INC SI
      INC DI
      DEC CX
      JNZ IS_EQUAL
      MOV DX,OFFSET MSG2
      MOV AH,9
      INT 21H
      MOV AH,4CH
      INT 21H
      
      INVALID:
      MOV DX,OFFSET MSG3
      MOV AH,9
      INT 21H
      MOV AH,4CH
      INT 21H
       
      INCREMENT:
      MOV [SI],AL
      JMP PROGRAM
      
  MAIN ENDP
END MAIN

\end{lstlisting}


\section{Write a program to reverse the given string for 8086.} 

\begin{lstlisting}[style=asmStyle]
.MODEL SMALL
.STACK 100H

.DATA
MSG DB 'KAZI OMAR SHARIF$'

.CODE
    MAIN PROC
        
        MOV AX, @DATA
        MOV DS, AX
        
        MOV SI, OFFSET MSG
        MOV CX, 16
        
        LOOP1:
            MOV BX, [SI]
            PUSH BX
            INC SI
            LOOP LOOP1
        
        
        MOV CX, 16
        
        LOOP2:
            POP DX
            MOV AH, 02H
            INT 21H
            LOOP LOOP2
        
        
        MAIN ENDP
    END MAIN

\end{lstlisting}

\section{Write a program in 8086 microprocessor to sort numbers in ascending order in an
array of n numbers, where size “n” is stored at memory address 2000 : 500 and
the numbers are stored from memory address 2000 : 501.} 

\begin{lstlisting}[style=asmStyle]
org 100h
.model small
.data
      msg1 db "Enter 10 number : $" 
      msg2 db "The sorted array is : $" 
      array db 10 ?
.code
main proc
      lea dx,msg1
      mov ah,9
      int 21h
     mov cx,10
      mov bx,offset array
      mov ah,1
      input:
      int 21h
      mov [bx],al
      inc bx
      loop input
      mov cx,10
      dec cx
      outer_loop:
      mov bx,cx
      mov si,0    
      inner_loop:
      mov al,array[si]
      mov dl,array[si+1]   
      cmp al,dl
      jc no_swap
      mov array[si], dl
      mov array[si+1], al
      no_swap: 
      inc si
      dec bx
      jnz inner_loop
      loop outer_loop
      mov ah,2
      mov dl,10
      int 21h
      mov dl,13
      int 21h
      lea dx,msg2
      mov ah,9
      int 21h
      mov cx,10
      mov bx,offset array
      output:
      mov dl,[bx]
      mov ah,2
      int 21h
      mov dl,32
      mov ah, 2
      int 21h
      inc bx
      loop output
main endp 
ret
\end{lstlisting}

%\section{Methodology}
%\lipsum[2-3] % Dummy text



% Bibliography (if needed)
% \bibliographystyle{plain}
% \bibliography{references}

\end{document}
